\documentclass{tufte-handout}

%\geometry{showframe}% for debugging purposes -- displays the margins
\usepackage{amsmath}

% Set up the images/graphics package
\usepackage{graphicx}
\setkeys{Gin}{width=\linewidth,totalheight=\textheight,keepaspectratio}
\graphicspath{{graphics/}}

\title{Carbohydrate Structure}
\author{Olivia Anderson}
\date{}  % if the \date{} command is left out, the current date will be used

% The following package makes prettier tables.  We're all about the bling!
\usepackage{booktabs}

% The units package provides nice, non-stacked fractions and better spacing
% for units.
\usepackage{units}

% The fancyvrb package lets us customize the formatting of verbatim
% environments.  We use a slightly smaller font.
\usepackage{fancyvrb}
\fvset{fontsize=\normalsize}

% Small sections of multiple columns
\usepackage{multicol}

% Provides paragraphs of dummy text
\usepackage{lipsum}

% These commands are used to pretty-print LaTeX commands
\newcommand{\doccmd}[1]{\texttt{\textbackslash#1}}% command name -- adds backslash automatically
\newcommand{\docopt}[1]{\ensuremath{\langle}\textrm{\textit{#1}}\ensuremath{\rangle}}% optional command argument
\newcommand{\docarg}[1]{\textrm{\textit{#1}}}% (required) command argument
\newenvironment{docspec}{\begin{quote}\noindent}{\end{quote}}% command specification environment
\newcommand{\docenv}[1]{\textsf{#1}}% environment name
\newcommand{\docpkg}[1]{\texttt{#1}}% package name
\newcommand{\doccls}[1]{\texttt{#1}}% document class name
\newcommand{\docclsopt}[1]{\texttt{#1}}% document class option name


\begin{document}
\maketitle% this prints the handout title, author, and date
\begin{abstract}
For this lecture we will review the basic structure of fiber and go over several definitions used by the general population. We will discuss the varying types of fibers and their characteristics. The characteristics of each fiber type are indication of how they affect digestive and absorptive processes. The digestive and absorptive effects of fiber can influence several health outcomes in humans.
\end{abstract}

\tableofcontents


\pagebreak

\section{Learning Objectives}

\begin{itemize}

\item Review structure of fiber
\item Compare various types of fibers
\item Describe the various properties of fiber
\item Apply the structural properties of fiber to physiological effects on humans
\end{itemize}

\section{Fiber Basics}\index{Fiber|(}
We discuss fiber within the carbohydrate unit because its structure is very similar except that the glycosidic bonds are resistant to digestive breakdown and will forgo absorption in the digestive tract. For example, recall the polysaccharide cellulose\index{Carbohydrates!Cellulose}. It is a linear homopolymer of beta-D glucose linked by beta 1, 4 bonds which are resistant to glycosidases in the digestive tract such as alpha amylase\index{Digestive Enzymes!Pancreatic Alpha Amylase} (enzyme specificity to alpha 1, 4 glycosidic bonds\index{Carbohydrates!Glycosidic Bonds}). This results in cellulose passing through the digestive tract intact and reaching the large intestine\index{Large Intestine}. Fiber is found in plants, specifically within the cellular walls. Common dietary sources of fiber include fruits, vegetables, grains and legumes. Fiber can also be synthetically added to food sources such as cereals, yogurt, juices, and even artificial sweeteners.

\section*{Defining Fiber}
The term crude fiber was coined by two scientists in the 1800’s who first discovered that there was material leftover from plants after an extraction in an acidic dilute followed by an alkali dilute which mimicked the digestive tract environment. As an understanding of how the digestive tract works and what this leftover fraction was, the definition of crude fiber evolved over time to what we most commonly refer to as dietary fiber or in other words, "the plant polysaccharides and lignin which are resistant to hydrolysis by the digestive enzymes of a man" \citep{Trowell1978}. For research and policy purposes, several definitions beyond dietary fiber have been developed by scientific and regulatory agencies. The definitions either encompass a physiological character of the fiber such as soluble\index{Fiber!Soluble Fiber} or fermentable\index{Fiber!Fermentable} or refer to an analytical method (i.e., can be synthetically made) associated with the fiber like functional fiber\index{Fiber!Functional Fiber}.

\section{Key Characteristics (and Physiological Effects)}

\subsection{Solubility}\index{Fiber!Soluble Fiber|(}
As the name of the characteristic suggests, fibers defined as water-soluble will dissolve in water whereas water-insoluble will not. The solubility of a fiber results in unique physiological effects. Fibers with a higher water solubility form a more gel-like substance as they move through the tract. They also tend to have a high viscosity, ability to adsorb and are typically fermentable\sidenote{See other characteristics sections below for more detail}. Alternatively, insoluble fibers will stay intact as they travel through the digestive tract. Upon reaching the colon insoluble fiber will add to the bulk of fecal matter decreasing its transit time through the large intestine. This property of insoluble fibers helps with the relief of constipation.\index{Fiber!Soluble Fiber|)}\index{Constipation}

\subsection{Fermentable}\index{Fiber!Fermentable|(}
Whether a fiber is fermentable or not depends on whether the bacteria in our large intestine can ferment it (\textit{i.e.}, metabolize it). Byproducts of fermentation include short-chained fatty acids\index{Lipids!Short-Chain Fatty Acids} (acetate, propionate, and butyrate), carbon dioxide and hydrogen. The short chained fatty acids that are produced as by-products can either be used by colon cells\index{Colonocytes} for their own energy provision or they can be absorbed at the large intestine\index{Large Intestine} entering general circulation and used for energy by non-colon tissue.\index{Fiber!Fermentable|)}

\subsection{Functionality}\index{Fiber!Functional Fiber|(}
The most widely used definition established by the Institute of Medicine is based on functionality \citep{InstituteofMedicine2005}. Dietary fibers are used specifically for nondigestible carbohydrates that are still intact and intrinsic in plant food sources. Dietary fibers encompass the naturally occurring fibers in food. Functional fibers are non-digestible carbohydrates that have been isolated from plant sources or synthesized in laboratory conditions - according to its definition, the isolated fiber alone has to provide benefits for human health. Thus, functional fibers encompass either natural fiber that has been separated from the original food or synthetic fiber produced by a human.\index{Fiber!Functional Fiber|)}

\subsection{Other Characteristics}
Water-soluble fibers are typically viscous. The viscosity turns the material more gel-like and slows the movement of food through the digestive tract. Viscous fibers delay gastric emptying leaving chyme in the stomach for a longer period of time and increase the time of feeling full \citep{Willis2009}\index{Satiety}\index{Hunger}. The slower gastric emptying\index{Gastric Emptying} can play a role in the rate of glucose absorption at the small intestine\index{Small Intestine} which aides in the well-controlled flux of glucose levels\index{Blood Glucose} into circulation following food intake. In addition to sequestering carbohydrates, viscous fibers can also sequester proteins and lipids inhibiting their exposure to digestive enzymes. This can impede absorption of these macronutrients at the small intestine. Some fibers have the ability to adsorb (\textit{i.e.}, to bind) to molecules and nutrients. Relevant to human health, some fibers bind fatty acids\index{Lipids!Free Fatty Acids}, cholesterol\index{Lipids!Cholesterol} and bile acids\index{Bile Salts} within the digestive tract. Once bound, the material travels to the large intestine, is added to the bulk of fecal matter and is excreted. If it does not get added to fecal matter, bacteria in the large intestine can metabolize the bound molecules. Focusing on the potential of increased bile excretion by fiber via feces\index{Feces}, the liver\index{Liver} will need to synthesize more bile to keep up with lipid digestion and absorption. Remember that part of the basic structure of bile includes cholesterol, thus LDL cholesterol\index{Apolipoproteins!LDL} will be taken from circulation and incorporated into bile ultimately decreasing serum cholesterol \citep{Brown1999}.

Although, this is a well-proposed mechanism, evidence shows that high intake of fiber (equivalent to >3 servings of oatmeal per day) on a regular basis is needed to result in a significant decrease in blood cholesterol. Fiber can also adsorb to specific minerals\index{Minerals} like calcium and iron. This can have either a positive or negative consequence. If the fiber is also highly fermentable\index{Fiber!Fermentable} with bound minerals, the breakdown of the fiber by gut bacteria\index{Gut Microbiota} will release the minerals and allow for additional mineral absorption at the large intestine (some minerals actually have efficient transport systems at the colonocytes\index{Colonocytes}). On the other hand, if the fiber is poorly fermentable, the minerals will remain intact with the fiber material and be incorporated into fecal matter.  \sidenote{\textbf{Reflection:} Use table \ref{tab:fiber-effects} to fill in the second column based on what you just read.}  

\begin{margintable}
\caption{Properties of fiber and their physiological effects}
\label{tab:fiber-effects}
\begin{tabular}{ll}
\hline
\textbf{Property} & \textbf{Physiological Effect} \\ \hline
Insoluble &  \\
Soluble &  \\
Fermentable &  \\
Viscous &  \\
Adsorbent &  \\ \bottomrule
\end{tabular}
\end{margintable}

\section{Types of Fiber}
The different types of fiber we find in our dietary sources vary greatly in structure and function. Please refer to Table \ref{tab:fiber-types} to see a list of common dietary fibers and what types of food sources they are found in, whether they can be synthesized for consumer products and the types of properties they hold.

% Please add the following required packages to your document preamble:
% \usepackage{booktabs}
\begin{table}[t]
\caption{Types of fiber and their properties.}\index{Fiber!Galactans}\index{Fiber!Fructans}\index{Fiber!Cellulose}\index{Fiber!Hemicellulose}\index{Fiber!Pectins}\index{Fiber!Gums}\index{Fiber!$\beta$-Glucans}
\label{tab:fiber-types}
\begin{tabular}{p{2cm}p{3cm}p{2cm}p{3cm}}
\toprule
\textbf{Fiber} & \textbf{Sources} & \textbf{Structure} & \textbf{Properties} \\ \midrule
Cellulose & Whole grains, root vegetables & beta-1,4 glucose & Insoluble, poorly fermented \\
Hemicellulose & Bran, nuts, legumes & Branched, various units & Depends on branching \\
Pectin & Fruits & Highly branched, various units & Soluble, fermentable, adsorbent \\
Gums & Oatmeal, barley, tree & Branched, various units & Soluble, fermentable, adsorbent \\
$\beta$-Glucans & Oatmeal, Rice & beta-1, 3 glucose, with branches & Soluble, fermentable \\
Fructans & Onions, Artichokes & polyfructose & Soluble, fermentable \\ 
Galactans & Chickpeas, Lentils & polygalactose & Soluble \\ \bottomrule
\end{tabular}
\end{table}


\index{Fiber|)}

\bibliography{library}
\bibliographystyle{plainnat}
\end{document}
